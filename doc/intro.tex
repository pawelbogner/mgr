\chapter{Introduction}
The problem of control of skid-steering mobile robots is a very complex issue. 
The main difficulty is to create an adequate model the phenomenon of friciton. 
The authors of \cite{caracciolo1999trajectory} use the Coulomb friction model limited to the low velocities.
Apart from that, the longitudinal skid was neglected. The approach presented in this paper is different.
The assumption is that the skids are possible in both directions, lateral and longitudinal.
This is similar to the models used in automotive literature \cite{pacejka2005tyre}.
All slips are allowed, but the impact of them to the model may be suppressed by tuning the
parameters properly. 

When it comes to the friction model idea, it is based on the velocity of the slip derived
from the Pfaffian matrix. %TODO cite 
Two types of models have been considered here: the linear and the discontinuous model.
The linear friction force model is analogous to the mass on the spring. In that case the friciton
force is proportional to the value of the slip, like the spring reaction force is proportional
to the displacement from the equilibrium position. The second model idea bases on the first one
but includes a modification in order to reflect the fact of losing the traction. Such change
consists in switching the friction coefficient dependently on the value on the slip.

Such systems may be considered on two levels --- kinematic and dynamic. The kinematic model may be
derived basing on the 

Motion planning plays an important part in robot control. This is usually the first step of solving
a certain problem. It is not enough to use a good control algorithm for path following. The fundamental
issue is determining the path or the trajectory to follow. There are many motion planning methods,
such as steering using sinusoid for chained systems \cite{murray1993nonholonomic} or rapidly-exploring random trees for nonholomic systems
with obstacles on the scene \cite{lavalle2000rapidly}. The planning algorithm presented in this paper,
the endogenous configuration approach, can be applied for mobile manipulators, regarding the
mobile platform on dynamic level and the kinematics of the manipulator arm.

The layout of this paper is as follows. Chapter \ref{ch:model} presents the models of researched
objects: the unicycle and RobRex mobile manipulator. Chapter \ref{ch:endogen} introduces the endogenous
configuration space approach to motion planning problem. In chapter \ref{ch:simul}, there are collected
the results of the simulations of the objects mentioned above with application of the motion
planning algorithm. Chapter \ref{ch:concl} summarizes the whole research made.