%\newgeometry{top=3cm}
\chapter{Introduction}
%\vspace{-0.7cm}
The problem of control of skid-steering mobile robots is a very complex issue. 
The main difficulty is to create an adequate model of the phenomenon of friciton. 
There has been many different attempts to describe it.
For instance, the authors of \cite{caracciolo1999trajectory} use the Coulomb friction model limited to the low velocities.
Apart from that, the longitudinal skid was neglected. The approach presented in this paper is different.
The assumption is that the slips are possible in both directions, lateral and longitudinal.
This is similar to the models used in automotive literature \cite{pacejka2005tyre}.
All slips are allowed, but the impact of them to the model may be suppressed by tuning the
parameters properly. 

Skid-steering platforms cover a wide class of mobile robots, including those with fixed-axis wheels. Such vehicles are able to turn only in presence of the slip. Such constructions are easier and cheaper to produce than those which wheels may be turned. Therefore, it is worth to analyse such systems and develop their models and control theory.

When it comes to the friction model idea used in this thesis,
it is based on the velocity of the slip derived
from the Pfaffian matrix. %TODO cite 
Two types of models have been considered here: the linear and the discontinuous model.
The linear friction force model is analogous to the mass on the spring. In that case the friciton
force is proportional to the value of the slip, like the spring reaction force is proportional
to the displacement from the equilibrium position. The second model idea bases on the first one,
but includes a modification in order to reflect the fact of losing the traction. Such a change
consists in switching the friction coefficient dependently on the value of the slip.

Skid-steering systems may be considered on two levels --- kinematic and dynamic.
The kinematic model may be derived basing on the motion constraints in the Pfaffian form, which may lead
to a nonholonomic system. This can be extended to include dynamics, using d'Alembert principle.

Motion planning plays an important part in robot control. This is usually the first step of solving
a certain control problem. It is not enough to use a good control algorithm for path following. The fundamental
issue is determining the path or the trajectory to follow. There are many motion planning methods,
such as steering using sinusoids for chained systems \cite{murray1993nonholonomic} or rapidly-exploring random trees for nonholomic systems
with obstacles on the scene \cite{lavalle2000rapidly}. The planning algorithm presented in this paper,
the endogenous configuration approach, can be applied for mobile manipulators, regarding the
mobile platform on dynamic level and the kinematics of the manipulator arm. It has been extensively
tested using RobRex mobile manipulator as a controlled model using different
friction coefficients configurations.

The thesis of this work is that endogenous configuration approach may be used for
motion planning of skid-steering mobile manipulators, including the dynamics of the platform.

The layout of this thesis is as follows. Chapter \ref{ch:model} presents the models of studied
objects: the unicycle and RobRex mobile manipulator. Chapter \ref{ch:endogen} introduces the endogenous
configuration space approach to motion planning problem. In chapter \ref{ch:simul}, there are collected
the results of the simulations of the objects mentioned above, with application of the motion
planning algorithm. Chapter \ref{ch:concl} summarizes the whole study.
%\restoregeometry