\chapter{Introduction}

The problem of control of skid-steering mobile robots 
There has been many attempts to model the phenomenon of friciton. 
The authors of \cite{caracciolo1999trajectory} use the Coulomb friction model limited to the low velocities.
Apart from that, the longitudinal skid was neglected. The approach presented in this paper is different.
The assumption is that the skids are possible in both directions, lateral and longitudinal.
This is similar to the models used in automotive literature \cite{pacejka2005tyre}.
All slips are allowed, but the impact of them to the model may be suppressed by tuning the
parameters properly. 

When it comes to the friction model idea, it is based on the velocity of the slip derived
from the Pfaffian matrix. %TODO cite 
Two types of models have been considered here: the linear and the discontinuous model.


The layout of this paper is as follows. Chapter \ref{ch:model} presents the models of researched
objects: the unicycle and RobRex mobile manipulator. Chapter \ref{ch:endogen} introduces the endogenous
configuration space approach to motion planning problem. In chapter \ref{ch:simul}, there are collected
the results of the simulations of the objects mentioned above with application of the motion
planning algorithm. Chapter \ref{ch:concl} summarizes the whole research made.