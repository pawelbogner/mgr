\chapter{Endogenous configuration space}
\label{ch:endogen}
\section{Basic concepts}
Let us consider a mobile manipulator --- a mobile platform equipped with a manipulator. Let $q = (w, \dot w)^T \in \mathbb{R}^{2n}$ denote the state of the platform, and $x \in \mathbb{R}^p$ denote the configuration of the manipulator. The platform is actuated by a control force vector $u \in \mathbb{R}^m$. We can define a vector $y \in \mathbb{R}^r$ which is the result of the output function $k: \mathbb{R}^{2n} \times \mathbb{R}^p \rightarrow \mathbb{R}^r$, $y=k(q,x)$. 

Suppose the dynamics of the platform are defined by the equation obtained using the Euler-Lagrange formalism:
\begin{equation}
Q(w)\ddot w + C(w, \dot{w})\dot{w}+D(w)=F(w, \dot w)+B(w)u,
\end{equation}
where $Q(w)$ is the inertia matrix, $C(w, \dot w)$ denotes the centripetal and Coriolis forces matrix,
$D(w)$ is the potential forces vector, $F(w, \dot w)$ is the vector of traction forces and $u$ is the
control forces vector applied to the system through the matrix $B(w)$.
Such system may be transformed into the form $\dot q = f(q) + G(q)u$ using
\begin{equation}
\begin{aligned}
f(q)&=\begin{bmatrix}
\dot{w}\\
Q^{-1}(w)\left(F(w, \dot w)-C(w, \dot{w})\dot{w}-D(w)\right)
\end{bmatrix}, & G(q)&=\begin{bmatrix}
0_{n\times m}\\
Q^{-1}(w)B(w)
\end{bmatrix}.
\end{aligned}
\end{equation}
Adding to the above the output function depending on the platform state and the configuration of the manipulator gives the following affine control system with inputs
$u \in \mathbb{R}^m$, $x \in \mathbb{R}^p$ and output function $k$
\begin{equation}
\begin{cases}
\begin{aligned}
\label{eq:control_sys}
\dot q &= f(q) + G(q)u,\\
y &= k(q, x).
\end{aligned}
\end{cases}
\end{equation}

\subsection{Endogenous configuration}
Define a pair $(u(\cdot), x)$, which consists of control inputs mentioned above.
Such elements belong to the endogenous configuration space
$\mathcal{X} = L_m^2[0, T] \times \mathbb{R}^p$ of the mobile manipulator, where $T$ denotes a
control horizon.
The inner product in $\mathcal{X}$ can be defined as follows \cite{ecs_ijc}
\begin{equation}
\langle (u_1(\cdot), x_1), (u_2(\cdot), x_2) \rangle = \int_0^T u_1^T(t) u_2(t) \ud t + x_1^T x_2
\end{equation}
The function
$K_{q_0, T}: \mathcal{X} \rightarrow \mathbb{R}^r$
allows to determine the state of the mobile manipulator at time $T>0$, given
the endogenous configuration $(u(\cdot), x)$ and the initial state $q_0$
\begin{equation}
\label{eq:endmap}
K_{q_0, T}((u(\cdot), x)) = k(\phi_{q_0, T}(u(\cdot)), x),
\end{equation}
where $\phi_{q_0, T}(u(\cdot))$ denotes the flow of the system \eqref{eq:control_sys}
caused by the input $u(\cdot)$. The map \eqref{eq:endmap} will be called the end-point map.

With the above definitions, the analytic Jacobian of $K_{q_0, T}$ at $(u(\cdot), x) \in \mathcal{X}$ can be defined as \cite{ratajczak2013multiple}
\begin{align}
\label{eq:jacobian}
J_{q_0, T}(u(\cdot), x)(v(\cdot), w) &= \left.\frac{\ud}{\ud \alpha}\right|_{\alpha=0} K_{q_0, T}((u(\cdot)+\alpha v(\cdot), x+\alpha w))\\
 &= 
 C(T,x)\int_0^T \Phi(T,s)B(s)v(s)\ud s + D(T,x)w,
\end{align}
where $(v(\cdot), w)\in \mathcal{X}$ and $\Phi(T,s)$ is the fundamental matrix of the linear system \cite{tchon2004acceleration}
\begin{equation}
\begin{aligned}
\label{eq:abxi}
\dot \xi &= A(t)\xi + B(t) v, \\
\eta &= C(t, x)\xi + D(t, x).
\end{aligned}
\end{equation}
The matrices in \eqref{eq:abxi} are defined as follows
\begin{equation}
\label{eq:matrices}
\begin{aligned}
A(t) &= \frac{\partial (f(q(t))+G(q(t))u(t)}{\partial q}, & C(t, x) &= \frac{\partial k(q(t), x)}{\partial q},\\
B(t) &= G(q(t)), & D(t, x) &= \frac{\partial k(q(t), x)}{\partial x},
\end{aligned}
\end{equation}
where $q(t)=\phi_{q_0, t}(u(\cdot))$.
\section{Motion planning algorithm}
Given the end-point map $K_{q_0, T}(u(\cdot), x)$ and a desired output $y_d\in\mathbb{R}$, the motion planning problem is equivalent to finding an endogenous
configuration $(u_d(\cdot), x_d)$, such that $K_{q_0, T}(u_d(\cdot), x_d)=y_d$. In order to find a
solution, a Jacobian algorithm may be employed, analogically to the algorithms
for the inverse kinematics problem for manipulators. 

Let us choose an initial configuration $(u_0(\cdot), x_0)$, and
define in $\mathcal{X}$ a curve $(u_\theta(\cdot), x_\theta)$, $\theta \in \mathbb{R}$.
Computing the error along $(u_\theta(\cdot), x_\theta)$, we obtain
\begin{equation}
\label{eq:endogen_err}
e(\theta)=K_{q_0, T}(u_\theta(\cdot), x_\theta)-y_d.
\end{equation}
Since we request the error to stabilize at $0$ exponentially fast with a decay rate
$\gamma>0$, we get the formula
\begin{equation}
\label{eq:endogen_decay}
\frac{\ud e(\theta)}{\ud \theta}=-\gamma e(\theta).
\end{equation}
Combining \eqref{eq:endogen_err} and \eqref{eq:endogen_decay} we acquire the Ważewski-Davidenko type equation
\begin{equation}
J_{q_0, T}(u_\theta(\cdot), x)\dfrac{\ud}{\ud \theta}\begin{pmatrix}
u_\theta(\cdot)\\ x_\theta
\end{pmatrix}=-\gamma e(\theta).
\end{equation}
Let $J^\#_{q_0, T}(u_\theta(\cdot), x)$ denote the Moore-Penrose inverse \cite{ecs_ijc}
of the Jacobian \eqref{eq:jacobian}. By applying it to the above equation we get a dynamic system
\begin{equation}
\label{eq:endogen_alg}
\dfrac{\ud}{\ud \theta}\begin{pmatrix}
u_\theta(\cdot)\\ x_\theta
\end{pmatrix}=-\gamma J^\#_{q_0, T}(u_\theta(\cdot), x_\theta) e(\theta),
\end{equation}
which defines a solution of the formulated motion planning problem in the following way
\begin{equation}
\begin{pmatrix}
u_d(\cdot)\\ x_d
\end{pmatrix}=\lim_{\theta\rightarrow\infty}\begin{pmatrix}
u_\theta(\cdot)\\ x_\theta
\end{pmatrix}.
\end{equation}

\section{Numerical computations}
The dimension of the $L^2[0,T]$ space is infinite. Therefore, we need an infinite number of parameters to define an element from this space. This is impossible to achieve in calculations made on a computer. The solution is to limit the bandwidth of considered control functions. 

It is possible to choose a finite-dimensional orthogonal base $\Psi = ( \psi_1, \psi_2, \dots, \psi_s )$
spanning the band-limited control function space. 
Using such an approach, a function $u_i(t)$ is represented by a vector 
$\lambda_i = (\lambda_{i1}, \lambda_{i2}, \dots, \lambda_{is})^T$ 
which satisfies the formula $u_i(t) = \sum_{k=1}^s \psi_k(t) \lambda_{ik}$.

In order to define a control vector $u=(u_1, u_2, \dots, u_m)^T$, a matrix $P(t)$ should be defined, such that $u_\lambda(t)=P(t)\lambda$. This means that $P(t)$ is a block diagonal matrix of the form
\begin{equation}
\label{eq:Pt}
P(t)=\begin{bmatrix}
\Psi & 0 & \cdots & 0\\
0 & \Psi &  & \vdots\\
\vdots &  & \ddots & 0 \\
0 &  \cdots & 0 & \Psi
\end{bmatrix}
\end{equation}
The above formulae allow us to define a band-limited Jacobian as \cite{ecs_ijc}
\begin{equation}
\hat J_{q_0, T}(\lambda, x)=\begin{bmatrix}C_\lambda(T,x)\int_0^T \Phi_\lambda(T,s)B_\lambda(s) P(s)\ud s
& D_\lambda(T,x)\end{bmatrix},
\end{equation}
where the matrices $B_\lambda(s)$, $C_\lambda(T,x)$, $D_\lambda(T,x)$ are those from \eqref{eq:matrices}
calcutated assuming 
Assuming the above, the Jacobian $\hat J_{q_0, T}(\lambda, x)$
may be computed as\\ $\begin{bmatrix}
C_\lambda(T,x)\Xi(T)& D_\lambda(T,x)
\end{bmatrix}$, where $\Xi$ is a matrix defined by  
\begin{equation}
\dot \Xi = A_\lambda(t)\Xi +B_\lambda(t)P(t)
\end{equation}
with the initial condition $\Xi(0)=0$. 
This allows to easily calculate the Jacobian simultaneously with the simulation
of the system by extending the function being integrated during ODE solving.

In order to acquire the Moore-Penrose inverse of the Jacobian we can simply use the formula
\begin{equation}
\hat J^\#_{q_0, T}(\lambda, x)=\hat J^T_{q_0, T}(\lambda, x)\left(\hat J_{q_0, T}(\lambda, x)\hat J^T_{q_0, T}(\lambda, x)\right)^{-1}.
\end{equation}
The problem which may occur here is the singularity of the dexterity matrix\\
$\mathcal{G}_{q_0,T}(\lambda,x)=\hat J_{q_0, T}(\lambda, x)\hat J^T_{q_0, T}(\lambda, x)$.
The inverse can be regularised by adding a small value $\kappa$ to the diagonal of the matrix to
be inverted, in the following way
\begin{equation}
\hat J^\#_{q_0, T}(\lambda, x)=\hat J^T_{q_0, T}(\lambda, x)\left(\hat J_{q_0, T}(\lambda, x)\hat J^T_{q_0, T}(\lambda, x)+\kappa I_r\right)^{-1}.
\end{equation}
The regularised version of the inverse Jacobian should be used, only if the determinant of the dexterity matrix is below a certain assumed threshold.

Another limitation met in numerical approach to the motion planning problem is the necessity to discretise the variable $\theta$ in \eqref{eq:endogen_alg}.
This may be resolved by applying the first-order Euler method
\begin{equation}
\label{eq:endogen_num}
\begin{pmatrix}
\lambda_{k+1}\\
x_{k+1}
\end{pmatrix}  =\begin{pmatrix}\lambda_{k}\\
x_{k}
\end{pmatrix} - \gamma \hat J^\#_{q_0, T}(\lambda_k, x_k)e(k),
\end{equation}
where $e(k)=K_{q_0, T}(u_{\lambda_k}(\cdot), x_k)-y_d$.

In order to find the numerical solution, a vast array of numerical differential
equation solving algorithms may be employed. These include those with fixed step
(e.g. Runge-Kutta methods), as well as variable step ones \cite{solvers}.