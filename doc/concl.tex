\chapter{Conclusions}
\label{ch:concl}
This work was intended to examine the performance of the motion planning algorithm basing on the endogenous configuration space approach applied to the skid-steering mobile robots. The objects used in the research were the unicycle and the mobile manipulator. The latter was investigated either as a separate mobile platform or as a mobile platform equipped with a manipulator.
The algorithm was used with continuous as well as discontinuous models. The discontinuities were introduced through switching friction coefficients dependently on the value of the slip. The unicycle and the mobile platform were considered on the dynamic level. As for the manipulator, only its kinematics were taken into account.

The motion planning approach relies on the Jacobian, similarly to the inverse kinematics algorithms, so there may occur problems with singularities. However, the algorithm behaved well for the continuous model of the mobile platform as well as for the mobile manipulator, and always produced a valid result. It has been noticed that the friction coefficients which define the slips configuration play an important role for the outcome of the computations. For a mobile robot with all fixed-axis wheels it is necessary to slip in the lateral direction in order to turn. For this reason, the results obtained with low lateral friction coefficients were better, as far as two issues are concerned. First of all, the total area needed to complete the parking manoeuvre was far less in these cases. Setting high friction coefficients makes the robot harder to turn and may even lead to singular configurations of the Jacobian, which makes the method impossible to use. This observation is reflected also by the value of the dexterity matrix determinant, which gets lower when lateral friction coefficients get higher. Second point are the controls parameters --- the energy and the amplitude. High lateral friction may require very high control torques, unobtainable by the control actuators.

The next aspect making the controls large is the control horizon. It has been observed that shortening it by two times results in the controls amplitude about three times greater. Similar effect occurs for the total energy of the control signal.

The research made in this work has also pointed to the another advantage of the endogenous configuration space approach. This algorithm may be used for planning the motion not only of the platform, but also of the end-effector. It has been shown that the control of the platform and the configuration of the manipulator may be determined simultaneously, what results in a desired end-effector configuration.

Apart from algorithm performance assessment for continuous models, the important part of this thesis is examining this planning method applied to those models which contain discontinuities in friction forces. The friction model employed here had to reflect losing the traction. If the value of the slip was outside a certain interval, the friction coefficient was switched to a low value. The simulations of such situations were implemented with the benefit of the event mechanism from MATLAB. Such an approach allows to stop the simulation when a discontinuity is detected, save the state of the object and then start the next simulation with a new switched model assuming the same initial conditions as the terminal ones from the previous computations.

The discontinuous models studied here were the unicycle and the mobile platform. Both of them contained the above-mentioned friction coefficients switching, and the discontinuities were induced by this mechanism. Although it was possible to find sets of parameters for which the planning algorithm converged, this approach does not work in general. When it comes to the unicycle, switching friction coefficients was done not more than by one order of magnitude. As for the mobile platform, the differences were even smaller. Friction coefficients stayed at the same order of magnitude. Using such configurations resulted in convergence of the motion planning algorithm. If the differences were greater than these described above, the algorithm did not converge.

That means that the cases presented in this work were rather special. Broadly speaking, the endogenous configuration space approach does not work with discontinuous models. The reason is the fundamental idea of this algorithm. It assumes that a small change to the controls induces a small change to the value of the output function. Therefore, we use the Jacobian method to determine the direction of the change applied to the endogenous configuration in the iteration of the algorithm, in order to move the object nearer to the desired state. Using this method, if the model contains discontinuities, such a modification of the current controls may lead in a wrong direction and cause the algorithm not to converge. An example of motion planning algorithm divergence was also presented in this thesis.

To summarise, motion planning with the endogenous configuration space approach may be used with a vast array of mobile robots, considering also their dynamics. These models may contain the slips of the vehicle. The considered objects can be also equipped with a manipulator. In such cases, the end configuration of the effector may be planned. When it comes to discontinuous models, the algorithm works well only in special cases and is divergent in general.