\chapter{Conclusions}
\label{ch:concl}
This work was intended to illustrate the performance of the motion planning algorithm
basing on the endogenous configuration space approach. This method is based on Jacobian,
similarly to the inverse kinematics algorithms, so there may occur problems with
singularity.
The research has been made 
using two models --- a unicycle and RobRex mobile manipulator. First of them was investigated
on a dynamical level and the latter was studied as a separate mobile platform and as the whole.

The algorithm used in this paper behaved well for the continuous model of the mobile platform
as well as for the mobile manipulator and always produced a valid result.
It has been noticed that the friction coefficients which
define the slips configuration play an important role for the outcome of the computations.
For a mobile robot with all fixed-axis wheels it is necessary to skid in the lateral direction
in order to turn. For this reason, the parameters of the results obtained with low lateral friction
coefficients were better. This has a few consequences. First of all, the total area needed to
complete the manoeuvre was far less in these cases. Setting high lateral friction
coefficients makes the robot harder to turn and may even lead to singular configurations.
This phenomenon is reflected also by the value of the dexterity matrix determinant.
Apart from this issue, the lateral friction coefficients have great impact to the values
of the control functions, regarding their energy and amplitude. High friction may cause
the input very high that it can become unobtainable for actuators.

The next aspect making the inputs large is the control horizon. It has been observed that shortening
it two times makes the amplitude of the input about three times greater. Similar effect is obsereved
for the total energy of the control signal.

The advantage of the algorithm is that it may be used for planning the motion with respect
to the effector position. It has been shown that the control of the platform and the configuration
of the manipulator may be determined simultaneously what results in a desired effector configuration.
The output function defining the desired state of the system may also include constraints
for the state of the platform unless it is physically feasible by the controlled object.

However, the important purpose of the research was to examine
this planning method applied to a discontinuous model.
In order to implement the simulations of the discontinuous model it is convenient to use the event
mechanism from MATLAB. It allows to stop the simulation when a discontinuity is detected, save the state
of the object and then start next simulation with a new switched model assuming the initial conditions
the same as the terminal ones from the previous computations.

The conclusion is that endogenous configuration approach does not work
in general, but some special cases may be found that the algorithm figures out useful results.
Broadly speaking, the result is positive when the changes at discontinuities are not
large. Two sets of friction coefficients has been tested and the algorithm
converged only in the case when the difference introduced by switching the model was little.
This is happening because the fundamental idea of the endogenous configurations approach
bases on the fact that small change to the control input induces a small change to the output function.
Assuming that we use Jacobian method to determine the direction of the input, we modify the endogenous
configuration in order to move the object nearer to the desired state. If the model contains
discontinuities, such modification may lead in the wrong direction and the algorithm does not converge.