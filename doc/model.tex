T\chapter{Model}
In this thesis two different models will be taken into consideration: the unicycle model with system dynamics and RobRex mobile manipulator with platform dynamics and arm kinematics.
\section{Unicycle}


\section{RobRex mobile manipulator}
The model consists of two parts: a mobile platform and a manipulator. The cart has got four fixed-axis wheels, which are actuated with two motors --- the wheels on both sides are coupled. The schematic structure of the vehicle is shown in . %TODO
The manipulator is mounted on the platform above the middle of the front wheel axis.
\subsection{Mobile platform}
Let $w\in \mathbb{R}^5$ be the configuration of the platform, where
$w=(x, y, a\phi, R\theta_1, R\theta_2)$.  
The dynamics model of the cart is analysed in the standard form used in robotics:
\begin{equation}
\label{eq:std_mdl}
P(w)\ddot w + D(w, \dot w) = F(w, \dot w) + Bu.
\end{equation}
The elaborate model derivation can be found in \cite{coupled}. 
The elements of \eqref{eq:std_mdl} are equal to
\begin{align*}
P(w) &= \begin{bmatrix}
Q_{11} & 0 & \frac{Q_{13}}{a} & 0 & 0\\
0 & Q_{22} & \frac{Q_{23}}{a} & 0 & 0\\
\frac{Q_{13}}{a} & \frac{Q_{23}}{a} & \frac{Q_{33}}{a} & 0 & 0\\
0 & 0 & 0 & \frac{Q_{44}}{R^2} & 0 \\
0 & 0 & 0 & 0 & \frac{Q_{55}}{R^2}
\end{bmatrix}, & 
D(w, \dot w) &= \frac{\dot w_3^2}{a^2}\begin{pmatrix}
-Q_{23} & Q_{13} & 0 & 0 & 0
\end{pmatrix}^T
\end{align*}
The components in the above equations are
\begin{align*}
Q_{11} = Q_{22} &= m_p+4m_w,\\
Q_{44} = Q_{55} &= 2I_{w33},\\
Q_{13} &= -m_p(a_{p1}\sin\frac{w_3}{a}+a_{p2}\cos\frac{w_3}{a})- 2m_wa\sin\frac{w_3}{a},\\
Q_{23} &=  m_p(a_{p1}\cos\frac{w_3}{a}-a_{p2}\sin\frac{w_3}{a})+ 2m_wa\cos\frac{w_3}{a},\\
Q_{33} &= I_{p33}+m_p(a_{p1}^2+a_{p2}^2)+4(I_{w11}+m_wb^2)+2m_wa^2.
\end{align*}

The meanings of used symbols are:
\begin{itemize}
\item $m_p$ ---platform mass;
\item $m_w$ ---one wheel mass;
\item $a_{p1}$, $a_{p2}$ --- coordinates of the platform's centre of mass with respect to the platform's coordinate system;
\item $I_{w11}$ --- wheel's moment of inertia with respect to X axis;
\item $I_{w33}$ --- wheel's moment of inertia with respect to Z axis;
\item $I_{p33}$ --- platform's moment of inertia with respect to Z axis.

\end{itemize}

Motion limitations of the platform may be written in Pfaffian form which is 
\begin{equation*}
H(w)=\begin{bmatrix}
-\sin\frac{w_3}{a} & \cos\frac{w_3}{a} & 0 & 0 & 0\\
-\sin\frac{w_3}{a} & \cos\frac{w_3}{a} & 1 & 0 & 0\\
 \cos\frac{w_3}{a} & \sin\frac{w_3}{a} & -\frac{b}{a} & -1 & 0\\
 \cos\frac{w_3}{a} & \sin\frac{w_3}{a} &  \frac{b}{a} &  0 & 1
\end{bmatrix} = \begin{bmatrix}
H^1(w)\\
H^2(w)\\
H^3(w)\\
H^4(w)
\end{bmatrix},
\end{equation*}
where $H^1$ corresponds to the lateral slip of front wheels, $H^2$ to the lateral slip of rear wheel, $H^3$ and $H^4$ depict the longitudinal slips of left and right wheels respectively.



\subsection{Manipulator arm}
The manipulator has got five rotational joints. Its structure is shown in. %TODO
Let $q\in \mathbb{R}^5$ denote the configuration of the arm and $a_2, \dots, a_5$ the lengths of the links. The individual transformation matrices in Denavit-Hartenberg convention are as follows:
\begin{align*}
A_0^1 &= \rot(Z, q_1)\rot(X, \frac{\pi}{2}),\\
A_1^2 &= \rot(Z, q_2)\tr (X, a_2)\rot(X, -\frac{\pi}{2}),\\
A_2^3 &= \rot(Z, q_3)\tr (X, a_3)\rot(X, \frac{\pi}{2}),\\
A_3^4 &= \rot(Z, q_4)\tr (X, a_4)\rot(X, -\frac{\pi}{2}),\\
A_4^5 &= \rot(Z, q_5)\tr (X, a_5).
\end{align*}
In order to reduce kinematics complexity and ease obtaining output function in Euler angles form it has been assumed that third joint is fixed to $0$. This results in the transformation matrix $
A_0^5=\begin{bmatrix}
R_0^5 & T_0^5\\
0 & 1
\end{bmatrix}$, 
where
\begin{align*}
T_0^5 &= \begin{pmatrix}
c_1\left((a_2+a_3)c_2 + c_{24}(a_4+a_5c_5)\right) - a_5s_1s_5\\
s_1\left((a_2+a_3)c_2 + c_{24}(a_4+a_5c_5)\right) + a_5c_1s_5\\
    (a_2+a_3)s_2 + s_{24}(a_4+a_5c_5)
\end{pmatrix},\\
R_0^5 &= \begin{bmatrix}
c_1c_{24}c_5-s_1s_5 & -s_1c_5-c_1c_{24}s_5 & -c_1s_{24}\\
s_1c_{24}c_5-c_1s_5 &  c_1c_5-s_1c_{24}s_5 & -s_1s_{24}\\
s_{24}c_5           & -s_{24}s_5           &  c_{24}
\end{bmatrix}.
\end{align*}
End effector orientation Euler angles (using $\rot(Z,\phi)\rot(X, \theta)\rot(Z, \psi)$) are: $\phi=q_1-\frac{\pi}{2}$, $\theta=q_2+q_4$, $\psi=q_5+\frac{\pi}{2}$.

