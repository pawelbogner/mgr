\chapter{Simulations}
This chapter presents the results obtained by simulations. The research has been made for two models
described in chapter \ref{ch:model}. The motion planning problem has been solved using endogenous
configuration space approach presented in chapter \ref{ch:endogen}.
\section{RobRex}
\subsection{Problem formulation}
The problem to be solved is to move the platform from the initial state
$q_0 = [w_0; \dot{w_0}] = (0, 0, a\frac{\pi}{2}, 0_7)^T$ to the desired state
$q_d = [w_d; \dot{w_d}] = (0, 0, a\frac{\pi}{2}, 0_7)^T$
within given amount of time $T$. When it comes to the manipulator,
its coordinates in task space $(\phi, \theta, \psi, y, z) $ should be equal to
$(-\frac{\pi}{2}, -\frac{\pi}{4}, \frac{\pi}{2}, 0, 0.2)$.
This corresponds to a parking manoeuvre with the manipulator
looking straight forward, observing the floor right in front of it.

Now two types of friction models will be discussed --- linear
and discontinuous. These models will be employed in simulations
regarding solving the above problem.

\subsection{Linear friction model}
\subsubsection{High high}

\subsubsection{High low}

\subsubsection{Low High}

\subsubsection{Low Low}

\subsection{Discontinuous friction model}
\section{Unicycle}